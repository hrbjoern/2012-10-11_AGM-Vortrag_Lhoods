%\documentclass[a4paper,11pt]{aastex}
\documentclass[a4paper,12pt]{article}

% Language & Encoding:
\usepackage[latin1]{inputenc}
%\usepackage[T1]{fontenc}
%\usepackage[ngerman]{babel}

% Math:
\usepackage{amssymb,amsmath,wasysym}

% Typefont:
\usepackage{lmodern}
\usepackage[urw-garamond]{mathdesign}
%\usepackage{txfonts}
%\usepackage{bookman}
%\usepackage{mathpazo}


% Graphics:
\usepackage{graphicx}
\usepackage{placeins} % allows \FloatBarrier

% Color:
\usepackage{color}

% Bibliography:
\usepackage[numbers,square]{natbib}   % ??
%\usepackage{natbib}   % ??
%\usepackage{plainnat}
%\usepackage{aas_macros}

% Misc:
\parindent0pt

% Captions:
\usepackage{caption}
\captionsetup{margin=10pt,font=small,textfont=it,labelfont=bf,format=hang}



% New commands: (Paper version)
\newcommand{\bit}{\begin{itemize}}
\newcommand{\eit}{\end{itemize}}
\newcommand{\beq}{\begin{equation}}
\newcommand{\eeq}{\end{equation}}
\newcommand{\bbl}{\begin{block}{}}
\newcommand{\ebl}{\end{block}}
\newcommand{\babl}{\begin{alertblock}{}}
\newcommand{\eabl}{\end{alertblock}}
\newcommand{\ra}{ \ensuremath{\longrightarrow} }
\newcommand{\Sun}{{\ensuremath{\tiny\astrosun}}}
\newcommand{\Eth}{ {\ensuremath{E_{\textrm{th}}}} }
\newcommand{\sigmav}{ {\ensuremath{\langle \sigma v \rangle}} }
\newcommand{\cms}{ {\ensuremath{\textrm{ cm}^3/\textrm{s}}} }
\newcommand{\jbar}{ {\ensuremath{\overline{J}} }}
\newcommand{\jbardo}{ {\ensuremath{\overline{J}(\Delta\Omega)} }}
\newcommand{\cbar}{ {\ensuremath{\overline{C}} }}
\newcommand{\gr}{{\ensuremath{\gamma \textrm{-ray}} }}
\newcommand{\grad}{\ensuremath{^{\textsf{o}}}}
\newcommand{\degr}{\ensuremath{^{\circ}} }
\newcommand{\gev}{ {\ensuremath{\textrm{ GeV}} }}
\newcommand{\gevcm}{ {\ensuremath{\textrm{ GeV}^2/\textsf{cm}^5} }}
\newcommand{\msun}{{\ensuremath{{\textrm M}_{\tiny\astrosun}}}}
\newcommand{\hess}{{H.E.S.S.}}
\newcommand{\wrt}{{with respect to} }
\newcommand{\dd}{\ensuremath{\mathrm{d}}}
\newcommand{\non}{\ensuremath{N_{\rm on}} }
\newcommand{\noff}{\ensuremath{N_{\rm off}} }
%\newcommand{\non}{\ensuremath{N_s}}
%\newcommand{\non}{\ensuremath{N_{\rm on}}}

\renewcommand{\L}{\ensuremath{\mathcal{L}} }


\title{Likelihoods etc.}
\author{Herr Bj�rn Opitz\footnote{bjoern.opitz@desy.de}}

\begin{document}
\maketitle
%\tableofcontents

\section{Preliminaries}

\subsection{Poissonian probability mass function:}
\beq 
P_p(x|\mu) = \frac{\mu^x}{x!}e^{-\mu}
\eeq
This is the probability to observe $x$ when the expectation value is
$\mu$. \\

\subsection{Likelihood function:}
It seems that usually, \L is defined as:
\beq 
\L(\mu|x) = P(x|\mu) ,
\eeq
that is, the \emph{likelihood} of a model expectation $\mu$ given the
data point $x$ is equal to the \emph{probability} of the data point
$x$ given the expectation value $\mu$. Important: In general, $\int \L
\neq 1$, \L is \emph{not} a PDF.

\clearpage

\section{Li \& Ma \& so on}

%\subsection{Maximum Likelihood:}

\subsection{Likelihood ratio testing:}
LR test statistic:
\beq
\Lambda(x) = \frac{\L(\theta_0|x)}{\L(\theta|x)} 
= \frac{\L(H_0)}{\L(H_1)}
\eeq
The null hypothesis $H_0$ is a special case of the alternative
hypothesis (-ses?) $H_1$, and hence has less degrees of
freedom. Therefore, $\L(H_1)$ is bound to be higher --- if the 
value of $\Lambda$ is too small, the null hypothesis can be rejected.

\subsubsection{Wilks' theorem:}
For large sample sizes, $-2 \ln \Lambda$ approaches a $\chi^2$
distribution with $N_{\rm dof}$ equal to the \emph{difference} in $N_{\rm dof}$
between $H_0$ and $H_1$.

\subsection{\hess-like Li\&Ma significance:}
Wilks' theorem can be used to calculate how \emph{significantly} the
signal disagrees with the null hypothesis ``no signal'' ($s = 0$).

Definition of the Likelihood functions: Data $X$, parameters $E,T$;
max. likelihood parameters $\hat{E},\hat{T}$, conditional max \L
parameter $\hat{T}_c$ (under the condition that $s=0$). Expectation
values $s,b$.
\begin{align}
\L_0(E_0,\hat{T}_c|X)  &= P(\non,\noff|s=0,b) \\
&= P(\non|\langle\non\rangle)\times P(\noff|\langle\noff\rangle) \\[10pt]
%\end{align}
%\begin{align}
\L_1(\hat{E},\hat{T}|X)  &= P(\non,\noff|s,b) \\
&= P(\non|\langle\non\rangle = \non)\times P(\noff|\langle\noff\rangle = \noff)
\end{align}

(See paper for $\langle\rangle$ definitions in $\L_0$! This is, I
guess, the main point in the Li\&Ma paper: They change a bit under the
assumption that $s=0$.) \\

\emph{NB:} Li\&Ma assume that the maximum likelihood values for the
data are $\langle\non\rangle = \non$ and $\langle\noff\rangle =
\noff$, for the calculation of $\L_1$. In principle, this likelihood
maximisation has to be performed first.

Using this, the significance $S$ can be calculated as 
\beq 
S = \sqrt{-2 \ln \Lambda}.
\eeq
See \texttt{Python/LiMa.py}.


\section{Parameter estimation}
We actually want to do something else: We want to estimate the
parameter~$s$ and possibly derive a confidence interval or an upper
limit to its value. 

For this, we consider $\L_1$  and maximise it / minimise $-2\ln \L$
\wrt  $s$: 
\begin{align}
\L_1(s,b|\non,\noff) &= P(\non|s,b)\times P(\noff|s,b) \\
&= P_p(\non|\langle\non\rangle=s+\alpha b) \times 
      P_p(\noff|\langle\noff\rangle=b) 
\end{align}
(\emph{Is this correct? Aren't $s$ and $\alpha b$ actually \emph{two}
  Poisson processes?}) 

Basically we have to calc. max \L \wrt both $s$ and $b$
  simultaneously. How would that work? Anyway:
\beq
\rightarrow \L_1(s) = _p(\non|\langle\non\rangle=s+\alpha \noff) \times 
      P_p(\noff|\langle\noff\rangle=\noff) , 
\eeq
assuming \noff as the best estimate for $b$. \\

How well does this work? Let's see: Assuming 
\[ s=20 \text{ or }0.0001, \alpha b = 200, b = 2000, \alpha = 0.1,\]
 the (modified) Likelihood function $-2 \ln\L(s)$  is shown in
 fig. \ref{fig:L1}. Also shown (as vertical lines) are
 the values where $-2 \ln\L(s) = -2 \ln \L_{\rm max} + 1$, i.e. where the
 $1\sigma$ error bars end. 

\begin{figure}[ht]
  \centering
  \includegraphics[width=0.45\textwidth]{2011-04-20_Nsgleich20.eps}
%  \caption{$s = 20$}
  \includegraphics[width=0.45\textwidth]{2011-04-20_Nsgleichnull.eps}
  \caption{left: $s = 20$, right: $s = 0.0001$}
  \label{fig:L1}
\end{figure}
% \begin{figure}[hb]
%   \centering
%   \includegraphics[width=0.4\textwidth]{2011-04-20_Nsgleichnull.eps}
%   \caption{$s = 0$}
%   \label{fig:L2}
% \end{figure}

\clearpage
For 10,000 simulations, the ``$s$ of maximum $\L$'' values are distributed
as shown in fig. \ref{fig:smin}:
\begin{figure}[h]
  \centering
  \includegraphics[width=0.7\textwidth]{2011-04-20_Gaussian_smin.eps}
  \caption{$s_{\rm min}$ distribution (normalized to area) for 10,000 simulated
    measurements, $\hat{s} = 0.001$.}
  \label{fig:smin}
\end{figure}


The distribution of $\sqrt{-2 \ln \Lambda(s)}$, i.e. what should look like a
$\sqrt{\chi^2(1)}$ distribution (really?), is shown in fig. \ref{fig:SL} together with
a one-sided normal Gaussian.
\begin{figure}[ht]
  \centering
  \includegraphics[width=0.45\textwidth]{2011-05-02_sqrtLambda_log.eps}
  \includegraphics[width=0.45\textwidth]{2011-05-02_sqrtLambda_lin.eps}
  \caption{ $\sqrt{-2\ln\Lambda(s)}$ distribution (normalized to area) for
    10,000 simulated   measurements, $\hat{s} = 0.0001$. Note: Left
    and right do \emph{not} show the same distribution.}
  \label{fig:SL}
\end{figure}

For comparison: Fig. \ref{fig:s20} shows the same distribution for
$\hat{s}=20$ events in 10,000 measurements. (Adding all measurements up
amounts  to 132 $\sigma$ significance!)
\begin{figure}[ht]
  \centering
  \includegraphics[width=0.7\textwidth]{2011-04-21_sqrtLambda_log_s20.eps}
  \caption{ $\sqrt{\Lambda(s)}$ distribution (normalized to area) for
    10,000 simulated   measurements, $\hat{s} = 20$}
  \label{fig:s20}
\end{figure}
%\clearpage
\FloatBarrier

\section{Combined Likelihood functions}

Generically: $i = 1...N$ measurements, 
\begin{align}
\L_{\rm tot} &= \prod_i \L_i \\
\ra -2\ln \L_{\rm tot} &= -2 \sum_i (\ln \L_i) \\
&\equiv \sum_i \mathtt{logLhood[i] \text{ in } Likelihood.py.}
\end{align}


\ra Fig. \ref{fig:CombL}: Plots f�r 10(00) Messungen, als Funktion von $s$ bei gemeinsamem
$\hat{s}$ 

\ra sp�ter auf \sigmav �bertragen!(?) 

\ra Noch sp�ter: \jbar samt Unsicherheiten ?

\begin{figure}[h]
  \centering
  \includegraphics[width=0.44\textwidth]{2011-05-02_CombLhood_neu_10-20.eps}
  \includegraphics[width=0.44\textwidth]{2011-05-02_CombLhood_neu_Garamond.eps}
  \includegraphics[width=0.44\textwidth]{2011-05-02_CombLhood_neu_10-0.eps}
  \includegraphics[width=0.44\textwidth]{2011-05-02_CombLhood_neu_1000-0.eps}
  \caption{Combined $\L(s)$ for $N = 10 / 1000$ ``measurements''
    simulated around the expectation value  $\hat{s} =  20 / 0.001$,
    each fig. showing the result of a single ``run''} 
  \label{fig:CombL}
\end{figure}
%\clearpage
\FloatBarrier

\section{What's next?}

\begin{enumerate}
  \item Was ist mit negativen Signifikanzen?
  \item $S$-Verteilung f�r 10 / 100 Messungen?
  \item Gesamt-$S$ f�r (1000$\times$ 10) Messungen?
  \item (Was ist denn nun $\chi^2$-verteilt, \L oder die likelihood
    ratio?) 
  \item (Was ist mit combined $\L$'s?)
  \item Wieso funzt (dort?) die Signifikanz-Bestimmung nicht?
  \item Wie �bertragen wir das dann auf \sigmav?
\end{enumerate}

% Bibliography:
%\bibliographystyle{amsplain}
%\bibliography{References.bib}

%\include{aas_macros}
%\bibliographystyle{apj}
%\bibliography{apj-jour.bib,References.bib}

%\printindex


\end{document}
